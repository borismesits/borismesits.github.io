\documentclass[25pt]{book}
\usepackage{amsmath}
\usepackage{fullpage}
\usepackage{graphicx}
\usepackage{ gensymb }
\usepackage{wasysym}
\usepackage{multicol}
\usepackage{braket}
\usepackage{dsfont}
\usepackage{url}
\usepackage{bm}
\usepackage{float}
\usepackage[margin=1in]{geometry}
\usepackage[colorlinks=true,linkcolor=blue,urlcolor=black,bookmarksopen=true]{hyperref}

\begin{document}
	
	\tableofcontents
	
	\chapter{Basic Principles}
	
	\section{Problem Types}
	
	\section{Derivations}
	
	\section{Concepts}
	
	\section{Equations}
	
	\subsection{Arbitrary Potential Oscillation Period}
	
	Period (as a function of energy) for an arbitrary potential energy $V$
	\[
	T(E)=\sqrt{2 m} \int_{x_{1}(E)}^{x_2\left(E_{2}\right)} \frac{d x}{\sqrt{E-U(x)}}
	\]
	with the turning points $x_1$ and $x_2$. 
	
	\subsection{Virial Theorem}
	
	\[
	\langle T\rangle=-\frac{1}{2} \sum_{k=1}^{N}\left\langle\mathbf{F}_{k} \cdot \mathbf{r}_{k}\right\rangle
	\]
	
	\subsection{Taylor's u-Equation}
	
	\[
	u^{\prime \prime}(\phi)=-u(\phi)-\frac{\mu}{\ell^{2} u(\phi)^{2}} F
	\]
	for a central force $F$, or, if you like
	\[
	F = \frac{-\partial V}{\partial r}
	\]
	
	\chapter{Accelerated Coordinate Systems}
	
	\section{Problem Types}
	
	\section{Derivations}
	
	\section{Concepts}
	
	\section{Equations}
	
	\subsection{General Fictitious Forces Eq.}
	
	Say we have a particle, which we can observe in an inertial frame $S$ and some other frame $S'$ (maybe non-inertial), called the \textit{body} frame, which is moving relative to the inertial frame. The position of $S'$ relative to $S$ is $\textbf{a}$. The angular velocity of $S'$ is $\bm{\omega}$ (same for both frames, as angular velocity does not depend on frame). The position of the particle in the body frame $S'$ is $\textbf{r}$ and is $\textbf{r}_0$ in $S$. Then
	\[
	m \left( \frac{d^2 \textbf{r}}{dt^2} \right)_{\text{body}} = m \left( \frac{d^2 \textbf{r}_0}{dt^2} \right)_{\text{inertial}}
	- m \left( \frac{d^2 \textbf{a}}{dt^2} \right)_{\text{inertial}} - 2 m \bm{\omega} \times \left( \frac{d \textbf{r}}{dt} \right) - m \bm{\omega} \times \left( \omega \times \textbf{r} \right) - m \frac{d\bm{\omega}}{dt} \times \textbf{r}
	\]
	Note that the $m \left( \frac{d^2 \textbf{r}_0}{dt^2} \right)_{\text{inertial}}$ term is just the external force in the inertial frame. That is, it is the term which represents the ``real'' forces. 
	
	\chapter{Lagrangian Dynamics}
	
	\section{Problem Types}
	
	\subsection{Find the Equilibrium Configuration}
	
	\textit{From Fetter and Walecka, P3.1:} Say a wire bent into a vertical circle with radius $a$ rotates at angular velocity $\Omega$, with gravity $\textbf{g}$, has a bead of mass $m$ free to move on the wire. 
	
	The Lagrangian would be
	\[
	\mathcal{L} = T - U = \frac{m}{2}\left( a^2 \dot{\theta}^2 + a^2 \sin^2 \theta \Omega^2  \right) + a \cos \theta mg
	\]
	and the one Euler-Lagrange equation would turn out to be
	\[
	\ddot{\theta} = \sin \theta \cos \theta \Omega^2 - \frac{g}{a} \sin \theta
	\]
	\textbf{To find the equilibrium position, we need to find a point where, if $\theta$ is stationary, then it remains stationary.} I.e., where $\ddot{\theta} = 0$. Thus we simply set
	\[
	0 = \sin \theta \cos \theta \Omega^2 - \frac{g}{a} \sin \theta
	\]
	And we find an equilibrium with condition
	\[
	\cos \theta_0 = \frac{g}{a \Omega^2}
	\]
	
	\subsection{find stability or frequency of oscillations about equilibrium}
	
	\subsection{Solve the Brachistochrone Problem}
	
	Just like we minimize the action
	\[
	S = \int \mathcal{L}(q, \dot{q}) dt
	\]
	equivalent to
	\[
	\frac{d}{dt} \frac{\partial \mathcal{L}}{\partial \dot{q}} - \frac{\partial \mathcal{L}}{\partial q}= 0
	\]
	we can minimize the time
	\[
	t_{12} = \int_1^{2} \frac{ds}{v}
	\]
	where $v$ is the velocity. Often we can use energy conservation to get
	\[
	v = \sqrt{2gy}
	\]
	where $y$ is the height, and if the particle is following a sloped path
	\[
	t_{12} = \int_1^{2} \sqrt{\frac{1 + (y')^2}{2gy} } dx
	\]
	Then writing
	\[
	\Phi = \left[ \frac{1 + (y')}{2gy} \right]^{1/2}
	\]
	the minimization condition is 
	\[
	\frac{d}{dx} \frac{\partial \Phi}{\partial y'} - \frac{\partial \Phi}{dy} = 0
	\]
	
	Note that earlier we could have changed our ``time-analogous'' coordinate and instead written
	changing the ``time-like'' variable
	\[
	t_{12} = \int_1^{2} \sqrt{\frac{1 + (x')^2}{2gy} } dy
	\]
	\section{Derivations}
	
	\subsection{Gauge Invariance of EM Lagrangian}
	
	For a charged particle in EM fields
	\[
	L(r,\dot{r},t) = \frac{1}{2} m \dot{r}^2 - q \phi(r, t) + q \dot{r} \cdot A(r,t)
	\]
	We apply the gauge transformation 
	\[
	A \rightarrow A + \nabla f
	\]
	\[
	\phi \rightarrow \phi - \frac{\partial f}{\partial t}
	\]
	We get
	\[
	L'(r,\dot{r},t) = \frac{1}{2} m \dot{r}^2 - q \phi(r, t) - q \frac{\partial f}{\partial t} + q \dot{r} \cdot A(r,t) + q \dot{r} \cdot \nabla f
	\]
	\[
	L'(r,\dot{r},t) = L(r,\dot{r},t)  - q \frac{\partial f}{\partial t}  + q \dot{r} \cdot \nabla f
	\]
	\[
	L'(r,\dot{r},t) = L(r,\dot{r},t)  - q \left(  \frac{\partial f}{\partial t}  +  \dot{r} \cdot \nabla f \right)
	\]
	\[
	L'(r,\dot{r},t) = L(r,\dot{r},t)  - q \left(  \frac{\partial }{\partial t}  +  \dot{r} \cdot \nabla \right)f
	\]
	The term in parenthesis is the convective derivative, which is equal to the total time derivative of $f$. See extra credit problem.
	\[
	L'(r,\dot{r},t) = L(r,\dot{r},t)  - q \frac{df}{dt}
	\]
	But from class we know that adding a total time derivative to the Lagrangian does not change the dynamics. 

	
	\section{Concepts}
	
	\section{Equations}
	

	
	
	
	\subsection{Euler-Lagrange Eq.'s}
	
	Euler-Lagrange Equations
	
	For one coordinate $q$
	\[
	\frac{d}{dt} \frac{\partial \mathcal{L}}{\partial \dot{q}} - \frac{\partial \mathcal{L}}{\partial q}= 0
	\]
	
	\subsection{Invariance of the Lagrangian}
	
	For the two Lagrangians 
	\[
	\mathcal{L} = T - V
	\]
	and 
	\[
	\mathcal{L}' = T - V + \frac{d f(x,t)}{dt}
	\]
	the dynamics are exactly the same for \textit{any} function $f(x,t)$. 
	
	
	
	
	
	\chapter{Small Oscillations}
	
	\section{Problem Types}
	
	\section{Derivations}
	
	\section{Concepts}
	
	\section{Equations}
	
	\chapter{Rigid Bodies}
	
	\section{Problem Types}
	
	\section{Derivations}
	
	\section{Concepts}
	
	\section{Equations}
	
	\subsection{Parallel Axis Theorem}
	\textbf{Also called Steiner's Theorem}
	
	Given the moment of inertia about the center of mass, this theorem allows us to calculate the moment of inertia about an axis offset from the center (although still pointing in the same direction). For center-of-mass MoI $I_c$, mass $M$, and axis offset $h$, the new moment of inertia is
	\[
	I = I_c + Mh^2
	\]
	
	\chapter{Hamiltonian Dynamics}
	
	\section{Problem Types}
	
	\section{Derivations}
	
	\section{Concepts}
	
	\section{Equations}
	
	\subsection{Hamilton-Jacobi Equation}
	
	For the Hamiltonian $\mathcal{H}$
	\[
	H
	\]
	
	\subsection{Hamilton's Eq.'s}
	
	
\end{document}
